\section{Integers}  
\subsection{Divisors}
%%%%%%%%%%%%%%%%%%%%%%%%%%%%%%%%%%%%%%%%%%%%%%%%%%%%%%%%%%%%%%%%%%%%%%%%%%%%%%%%
%%%%%%%%%%%%%%%%%%%%%%%%%%%%%%%%%%%%%%%%%%%%%%%%%%%%%%%%%%%%%%%%%%%%%%%%%%%%%%%%
%%%%%%%%%%%%%%%%%%%%%%%%%%%%%%%%%%%%%%%%%%%%%%%%%%%%%%%%%%%%%%%%%%%%%%%%%%%%%%%%
%%%%%%%%%%%%%%%%%%%%%%%%%%%%%%%%%%%%%%%%%%%%%%%%%%%%%%%%%%%%%%%%%%%%%%%%%%%%%%%%
\begin{mdframed}[style=darkQuesion]
1.    Let $m,n.r.s\in \Z$. If $m^2+n^2=r^2+s^2=mr+ns$, prove that $m=r$ and 
$n=s$.
\end{mdframed}

%%%%%%%%%%%%%%%%%%%%%%%%%%%%%%%%%%%%%%%%%%%%%%%%%%%%%%%%%%%%%%%%%%%%%%%%%%%%%%%%
\begin{mdframed}[style=darkAnswer,frametitle={Joe Starr}]
We select $m,n.r.s\in \Z$, given $m^2+n^2=r^2+s^2=mr+ns$ which can write
as $m^2+n^2-mr-ns=r^2+s^2-mr-ns$. From here we can simplify:
\begin{align*}
m^2+n^2-mr-ns=r^2+s^2-mr-ns &\Rightarrow 
m\lrp{m-r}+n\lrp{n-s}=r\lrp{r-m}+s\lrp{s-n} \\
&\Rightarrow m\lrp{m-r}+n\lrp{n-s}-r\lrp{r-m}-s\lrp{s-n}=0 \\
&\Rightarrow m\lrp{m-r}+r\lrp{m-r}+n\lrp{n-s}+s\lrp{n-s}=0 \\
&\Rightarrow \lrp{m-r}\lrp{m+r}+\lrp{n-s}\lrp{n+s}=0 \\
\end{align*}
from here we can see that in order for $\lrp{m-r}\lrp{m+r}+\lrp{n-s}\lrp{n+s}=0$
to be true $m=r$ and $n=s$.
\end{mdframed}
\newpage
%%%%%%%%%%%%%%%%%%%%%%%%%%%%%%%%%%%%%%%%%%%%%%%%%%%%%%%%%%%%%%%%%%%%%%%%%%%%%%%%
%%%%%%%%%%%%%%%%%%%%%%%%%%%%%%%%%%%%%%%%%%%%%%%%%%%%%%%%%%%%%%%%%%%%%%%%%%%%%%%%
%%%%%%%%%%%%%%%%%%%%%%%%%%%%%%%%%%%%%%%%%%%%%%%%%%%%%%%%%%%%%%%%%%%%%%%%%%%%%%%%
%%%%%%%%%%%%%%%%%%%%%%%%%%%%%%%%%%%%%%%%%%%%%%%%%%%%%%%%%%%%%%%%%%%%%%%%%%%%%%%%
\begin{mdframed}[style=darkQuesion]
3.    Find the quotient and reminder when $a$ id divided by $b$. 
\begin{itemize}
    \item [a] {$a=99$, $b=17$}
    \item [b] {$a=-99$, $b=17$}
    \item [c] {$a=17$, $b=99$}
    \item [d] {$a=-1017$, $b=99$}
\end{itemize}

\end{mdframed}

%%%%%%%%%%%%%%%%%%%%%%%%%%%%%%%%%%%%%%%%%%%%%%%%%%%%%%%%%%%%%%%%%%%%%%%%%%%%%%%%
\begin{mdframed}[style=darkAnswer,frametitle={Joe Starr}]
\begin{itemize}
    \item [a] {$99=17q+r\Rightarrow q=5, r=14$}
    \item [b] {$-99=17q+r\Rightarrow q=-6, r=3$}
    \item [c] {$17=99q+r\Rightarrow q=0, r=17$}
    \item [d] {$-1017=99q+r\Rightarrow q=-11, r=72$}
\end{itemize}
\end{mdframed}

\newpage
%%%%%%%%%%%%%%%%%%%%%%%%%%%%%%%%%%%%%%%%%%%%%%%%%%%%%%%%%%%%%%%%%%%%%%%%%%%%%%%%
%%%%%%%%%%%%%%%%%%%%%%%%%%%%%%%%%%%%%%%%%%%%%%%%%%%%%%%%%%%%%%%%%%%%%%%%%%%%%%%%
%%%%%%%%%%%%%%%%%%%%%%%%%%%%%%%%%%%%%%%%%%%%%%%%%%%%%%%%%%%%%%%%%%%%%%%%%%%%%%%%
%%%%%%%%%%%%%%%%%%%%%%%%%%%%%%%%%%%%%%%%%%%%%%%%%%%%%%%%%%%%%%%%%%%%%%%%%%%%%%%%
\begin{mdframed}[style=darkQuesion]
5.    Use the Euclidean algorithm to find the following greatest common divisors
\begin{itemize}
    \item [a] {$\lrp{6643,2873}$}
    \item [b] {$\lrp{7684,4148}$}
    \item [c] {$\lrp{26460,12600}$}
    \item [d] {$\lrp{6540,1206}$}
    \item [e] {$\lrp{12091,8439}$}
\end{itemize}

\end{mdframed}

%%%%%%%%%%%%%%%%%%%%%%%%%%%%%%%%%%%%%%%%%%%%%%%%%%%%%%%%%%%%%%%%%%%%%%%%%%%%%%%%
\begin{mdframed}[style=darkAnswer,frametitle={Joe Starr}]
\begin{itemize}
    \item [(a)] {$\lrp{6643,2873}$
    \begin{align*}
        6643&=2873*2+897\\
        2873&=897*3+182\\
        897&=182*4+169\\
        182&=169*1+13\\
        169&=13*13\\
    \end{align*}
    }
    \item [(b)] {$\lrp{7684,4148}$
    \begin{align*}
        7684&=4148*1+3536\\
        4148&=3536*1+612\\
        3536&=612*5+476\\
        612&=476*1+136\\
        476&=136*3+68\\
        136&=68*68\\
    \end{align*}
    }
    \item [(c)] {$\lrp{26460,12600}$
    \begin{align*}
        26460&=12600*2+1260\\
        12600&=1260*10\\
    \end{align*}
    }
    \item [(d)] {$\lrp{6540,1206}$
    \begin{align*}
        6540&=1206*5+510\\
        1206&=510*2+186\\
        510&=186*2+138\\
        186&=138*1+48\\
        138&=48*2+42\\
        48&=42*1+6\\
        42&=6*7\\
    \end{align*}
    }
    \item [(e)] {$\lrp{12091,8439}$
    \begin{align*}
        12091&=8439*1+3652\\
        8439&=3652*2+1135\\
        3652&=1135*3+247\\
        1135&=247*4+147\\
        247&=147*1+100\\
        147&=100*1+47\\
        100&=47*2+6\\
        47&=6*7+5\\
        6&=5*1+1\\
        5&=1*5\\
    \end{align*}
    }
\end{itemize}
\end{mdframed}
\newpage
%%%%%%%%%%%%%%%%%%%%%%%%%%%%%%%%%%%%%%%%%%%%%%%%%%%%%%%%%%%%%%%%%%%%%%%%%%%%%%%%
%%%%%%%%%%%%%%%%%%%%%%%%%%%%%%%%%%%%%%%%%%%%%%%%%%%%%%%%%%%%%%%%%%%%%%%%%%%%%%%%
%%%%%%%%%%%%%%%%%%%%%%%%%%%%%%%%%%%%%%%%%%%%%%%%%%%%%%%%%%%%%%%%%%%%%%%%%%%%%%%%
%%%%%%%%%%%%%%%%%%%%%%%%%%%%%%%%%%%%%%%%%%%%%%%%%%%%%%%%%%%%%%%%%%%%%%%%%%%%%%%%
\begin{mdframed}[style=darkQuesion]
    7.    For each part of Exercise 5, find integers $m$ and $n$ such that 
$\lrp{a,b}$ is expressed in the form $ma+nb$. 

\end{mdframed}

%%%%%%%%%%%%%%%%%%%%%%%%%%%%%%%%%%%%%%%%%%%%%%%%%%%%%%%%%%%%%%%%%%%%%%%%%%%%%%%%
\begin{mdframed}[style=darkAnswer,frametitle={Joe Starr}]
\begin{itemize}
    \item [(a)] {$\lrp{6643,2873}$\\
        $\lrp{6643}-16+\lrp{2873}37=13$
    }
    \item [(b)] {$\lrp{7684,4148}$\\
    $\lrp{7684}27+\lrp{4148}-50=68$
    }
    \item [(c)] {$\lrp{26460,12600}$
    $\lrp{26460}1+\lrp{12600}-2=1260$
    }
    \item [(d)] {$\lrp{6540,1206}$
   $\lrp{6540}-26+\lrp{1206}141=6$
    }
    \item [(e)] {$\lrp{12091,8439}$
    $\lrp{12091}1435+\lrp{8439}-2056=1$
    }
\end{itemize}
\end{mdframed}
\newpage
%%%%%%%%%%%%%%%%%%%%%%%%%%%%%%%%%%%%%%%%%%%%%%%%%%%%%%%%%%%%%%%%%%%%%%%%%%%%%%%%
%%%%%%%%%%%%%%%%%%%%%%%%%%%%%%%%%%%%%%%%%%%%%%%%%%%%%%%%%%%%%%%%%%%%%%%%%%%%%%%%
%%%%%%%%%%%%%%%%%%%%%%%%%%%%%%%%%%%%%%%%%%%%%%%%%%%%%%%%%%%%%%%%%%%%%%%%%%%%%%%%
%%%%%%%%%%%%%%%%%%%%%%%%%%%%%%%%%%%%%%%%%%%%%%%%%%%%%%%%%%%%%%%%%%%%%%%%%%%%%%%%
\begin{mdframed}[style=darkQuesion]
9.  let $a,b,c$ be integers such that $a+b+c=0$. Show that if $n$ is an integer
which is a divisor of two of the three integers, then it is also a divisor of 
the third. 

\end{mdframed}

%%%%%%%%%%%%%%%%%%%%%%%%%%%%%%%%%%%%%%%%%%%%%%%%%%%%%%%%%%%%%%%%%%%%%%%%%%%%%%%%
\begin{mdframed}[style=darkAnswer,frametitle={Joe Starr}]
Select $a,b,c\in \Z$ to satisfy $a+b+c=0$, WLOG let $n\in \Z$ such that 
$n\vert a$ and $n\vert b$. Since $\lrp{a+b}+c=0$ it must be that $\lrp{a+b}=-c$.
From here we must show $n\vert\lrp{a+b}$, or $a+b=nq$. Since $n\vert a$ and 
$n\vert b$ we may write $a=nq_1$ and $b=nq_2$, yielding, 
$nq_1+nq_2=n\lrp{q_1+q_2}=nq$ thus $n\vert c$, as desired.$\square$ 
\end{mdframed}
\newpage
%%%%%%%%%%%%%%%%%%%%%%%%%%%%%%%%%%%%%%%%%%%%%%%%%%%%%%%%%%%%%%%%%%%%%%%%%%%%%%%%
%%%%%%%%%%%%%%%%%%%%%%%%%%%%%%%%%%%%%%%%%%%%%%%%%%%%%%%%%%%%%%%%%%%%%%%%%%%%%%%%
%%%%%%%%%%%%%%%%%%%%%%%%%%%%%%%%%%%%%%%%%%%%%%%%%%%%%%%%%%%%%%%%%%%%%%%%%%%%%%%%
%%%%%%%%%%%%%%%%%%%%%%%%%%%%%%%%%%%%%%%%%%%%%%%%%%%%%%%%%%%%%%%%%%%%%%%%%%%%%%%%
\begin{mdframed}[style=darkQuesion]
13.  Show that if $n$ is any integer, then $\lrp{10n_3,5n+2}=1$
\end{mdframed}

%%%%%%%%%%%%%%%%%%%%%%%%%%%%%%%%%%%%%%%%%%%%%%%%%%%%%%%%%%%%%%%%%%%%%%%%%%%%%%%%
\begin{mdframed}[style=darkAnswer,frametitle={Joe Starr}]
We begin with the Euclidean algorithm, 
\begin{align*}
    10n+3&=\lrp{5n+2}1+\lrp{5n+1}\\
    5n+2&=\lrp{5n+1}1+1\\
\end{align*}
from here we have $\lrp{10n+3,5n+2}=\lrp{5n+2,5n+1}=1$, as desired.
\end{mdframed}
\newpage
%%%%%%%%%%%%%%%%%%%%%%%%%%%%%%%%%%%%%%%%%%%%%%%%%%%%%%%%%%%%%%%%%%%%%%%%%%%%%%%%
%%%%%%%%%%%%%%%%%%%%%%%%%%%%%%%%%%%%%%%%%%%%%%%%%%%%%%%%%%%%%%%%%%%%%%%%%%%%%%%%
%%%%%%%%%%%%%%%%%%%%%%%%%%%%%%%%%%%%%%%%%%%%%%%%%%%%%%%%%%%%%%%%%%%%%%%%%%%%%%%%
%%%%%%%%%%%%%%%%%%%%%%%%%%%%%%%%%%%%%%%%%%%%%%%%%%%%%%%%%%%%%%%%%%%%%%%%%%%%%%%%
\begin{mdframed}[style=darkQuesion]
15.  For what positive integers $n$ is it true that $\lrp{n,n+2}=2$? Prove your
claim.
\end{mdframed}

%%%%%%%%%%%%%%%%%%%%%%%%%%%%%%%%%%%%%%%%%%%%%%%%%%%%%%%%%%%%%%%%%%%%%%%%%%%%%%%%
\begin{mdframed}[style=darkAnswer,frametitle={Joe Starr}]
The conjecture is that the statement is true for even values of $n$.
We begin with rewriting $n$ in terms of $k$, $n=2k$the Euclidean algorithm, 
\begin{align*}
    \lrp{2k}+2&=\lrp{2k}1+\lrp{2}\\
    2k&=\lrp{2}{k}\\
\end{align*}
from here we have $\lrp{n+2,n}=\lrp{2k+2,2k}=2$, as desired.
\end{mdframed}
\newpage
%%%%%%%%%%%%%%%%%%%%%%%%%%%%%%%%%%%%%%%%%%%%%%%%%%%%%%%%%%%%%%%%%%%%%%%%%%%%%%%%
%%%%%%%%%%%%%%%%%%%%%%%%%%%%%%%%%%%%%%%%%%%%%%%%%%%%%%%%%%%%%%%%%%%%%%%%%%%%%%%%
%%%%%%%%%%%%%%%%%%%%%%%%%%%%%%%%%%%%%%%%%%%%%%%%%%%%%%%%%%%%%%%%%%%%%%%%%%%%%%%%
%%%%%%%%%%%%%%%%%%%%%%%%%%%%%%%%%%%%%%%%%%%%%%%%%%%%%%%%%%%%%%%%%%%%%%%%%%%%%%%%
\begin{mdframed}[style=darkQuesion]
17.  Show that the positive integer $k$ is the difference of two odd squares if 
and only if $k$ is divisible by $8$.
\end{mdframed}

%%%%%%%%%%%%%%%%%%%%%%%%%%%%%%%%%%%%%%%%%%%%%%%%%%%%%%%%%%%%%%%%%%%%%%%%%%%%%%%%
\begin{mdframed}[style=darkAnswer,frametitle={Joe Starr}]

We begin by writing $k=a^2-b^2$, since $a$ and $b$ are odd we can write, 
\begin{align*}
    a&=2r+1\\
    b&=2s+1
\end{align*} 
from here we have $q^2-b^2=4\lrp{r+s+1}\lrp{r-s}$. Since $k>0$ we must consider 
two cases $r-s=2m+1$ and $r-s=2m$. 
\begin{itemize}[align=left]
    \item [$r-s=2m$:]{\hspace{.5in}\newline
        In this case we have $q^2-b^2=4\lrp{r+s+1}2m=8\lrp{r+s+1}m$ and we are 
        done. 
    }
    \item [$r-s=2m+1$:]{\hspace{.5in}\newline
        In this case we have $r-s=2m+1$ and $r+s=r-s+2s=2m+1+2s$
        \begin{align*}
            q^2-b^2&=4\lrp{r+s+1}\lrp{2m+1}\\
            &=4\lrp{2m\lrp{r+s+1}+\lrp{r+s+1}}\\            
            &=4\lrp{\lrp{2mr+2ms+2m}+\lrp{r+s+1}}\\            
            &=4\lrp{2mr+2ms+2m+r+s+1}\\         
            &=4\lrp{2mr+2ms+2m+2m+1+2s+1}\\         
            &=4\lrp{2mr+2ms+2m+2m+2s+2}\\         
            &=8\lrp{mr+ms+m+m+s+1}\\         
        \end{align*}
        as desired.
    }
\end{itemize}

\end{mdframed}