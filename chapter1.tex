\section{Integers}  
\subsection{Divisors}
%%%%%%%%%%%%%%%%%%%%%%%%%%%%%%%%%%%%%%%%%%%%%%%%%%%%%%%%%%%%%%%%%%%%%%%%%%%%%%%%
%%%%%%%%%%%%%%%%%%%%%%%%%%%%%%%%%%%%%%%%%%%%%%%%%%%%%%%%%%%%%%%%%%%%%%%%%%%%%%%%
%%%%%%%%%%%%%%%%%%%%%%%%%%%%%%%%%%%%%%%%%%%%%%%%%%%%%%%%%%%%%%%%%%%%%%%%%%%%%%%%
%%%%%%%%%%%%%%%%%%%%%%%%%%%%%%%%%%%%%%%%%%%%%%%%%%%%%%%%%%%%%%%%%%%%%%%%%%%%%%%%
\begin{mdframed}[style=darkQuesion]
1.    Let $m,n.r.s\in \Z$. If $m^2+n^2=r^2+s^2=mr+ns$, prove that $m=r$ and 
$n=s$.
\end{mdframed}

%%%%%%%%%%%%%%%%%%%%%%%%%%%%%%%%%%%%%%%%%%%%%%%%%%%%%%%%%%%%%%%%%%%%%%%%%%%%%%%%
\begin{mdframed}[style=darkAnswer,frametitle={Joe Starr}]
We select $m,n.r.s\in \Z$, given $m^2+n^2=r^2+s^2=mr+ns$ which can write
as $m^2+n^2-mr-ns=r^2+s^2-mr-ns$. From here we can simplify:
\begin{align*}
m^2+n^2-mr-ns=r^2+s^2-mr-ns &\Rightarrow 
m\lrp{m-r}+n\lrp{n-s}=r\lrp{r-m}+s\lrp{s-n} \\
&\Rightarrow m\lrp{m-r}+n\lrp{n-s}-r\lrp{r-m}-s\lrp{s-n}=0 \\
&\Rightarrow m\lrp{m-r}+r\lrp{m-r}+n\lrp{n-s}+s\lrp{n-s}=0 \\
&\Rightarrow \lrp{m-r}\lrp{m+r}+\lrp{n-s}\lrp{n+s}=0 \\
\end{align*}
from here we can see that in order for $\lrp{m-r}\lrp{m+r}+\lrp{n-s}\lrp{n+s}=0$
to be true $m=r$ and $n=s$.
\end{mdframed}
\newpage
%%%%%%%%%%%%%%%%%%%%%%%%%%%%%%%%%%%%%%%%%%%%%%%%%%%%%%%%%%%%%%%%%%%%%%%%%%%%%%%%
%%%%%%%%%%%%%%%%%%%%%%%%%%%%%%%%%%%%%%%%%%%%%%%%%%%%%%%%%%%%%%%%%%%%%%%%%%%%%%%%
%%%%%%%%%%%%%%%%%%%%%%%%%%%%%%%%%%%%%%%%%%%%%%%%%%%%%%%%%%%%%%%%%%%%%%%%%%%%%%%%
%%%%%%%%%%%%%%%%%%%%%%%%%%%%%%%%%%%%%%%%%%%%%%%%%%%%%%%%%%%%%%%%%%%%%%%%%%%%%%%%
\begin{mdframed}[style=darkQuesion]
3.    Find the quotient and reminder when $a$ id divided by $b$. 
\begin{itemize}
    \item [a] {$a=99$, $b=17$}
    \item [b] {$a=-99$, $b=17$}
    \item [c] {$a=17$, $b=99$}
    \item [d] {$a=-1017$, $b=99$}
\end{itemize}

\end{mdframed}

%%%%%%%%%%%%%%%%%%%%%%%%%%%%%%%%%%%%%%%%%%%%%%%%%%%%%%%%%%%%%%%%%%%%%%%%%%%%%%%%
\begin{mdframed}[style=darkAnswer,frametitle={Joe Starr}]
\begin{itemize}
    \item [a] {$99=17q+r\Rightarrow q=5, r=14$}
    \item [b] {$-99=17q+r\Rightarrow q=-6, r=3$}
    \item [c] {$17=99q+r\Rightarrow q=0, r=17$}
    \item [d] {$-1017=99q+r\Rightarrow q=-11, r=72$}
\end{itemize}
\end{mdframed}

\newpage
%%%%%%%%%%%%%%%%%%%%%%%%%%%%%%%%%%%%%%%%%%%%%%%%%%%%%%%%%%%%%%%%%%%%%%%%%%%%%%%%
%%%%%%%%%%%%%%%%%%%%%%%%%%%%%%%%%%%%%%%%%%%%%%%%%%%%%%%%%%%%%%%%%%%%%%%%%%%%%%%%
%%%%%%%%%%%%%%%%%%%%%%%%%%%%%%%%%%%%%%%%%%%%%%%%%%%%%%%%%%%%%%%%%%%%%%%%%%%%%%%%
%%%%%%%%%%%%%%%%%%%%%%%%%%%%%%%%%%%%%%%%%%%%%%%%%%%%%%%%%%%%%%%%%%%%%%%%%%%%%%%%
\begin{mdframed}[style=darkQuesion]
5.    Use the Euclidean algorithm to find the following greatest common divisors
\begin{multicols}{2}
    \begin{itemize}
        \item [a] {$\lrp{6643,2873}$}
        \item [b] {$\lrp{7684,4148}$}
        \item [c] {$\lrp{26460,12600}$}
        \item [d] {$\lrp{6540,1206}$}
        \item [e] {$\lrp{12091,8439}$}
    \end{itemize}
\end{multicols}
\end{mdframed}

%%%%%%%%%%%%%%%%%%%%%%%%%%%%%%%%%%%%%%%%%%%%%%%%%%%%%%%%%%%%%%%%%%%%%%%%%%%%%%%%
\begin{mdframed}[style=darkAnswer,frametitle={Joe Starr}]
\begin{multicols}{2}
\begin{itemize}
    \item [(a)] {$\lrp{6643,2873}$
    \begin{align*}
        6643&=2873*2+897\\
        2873&=897*3+182\\
        897&=182*4+169\\
        182&=169*1+13\\
        169&=13*13\\
    \end{align*}
    }
    \item [(b)] {$\lrp{7684,4148}$
    \begin{align*}
        7684&=4148*1+3536\\
        4148&=3536*1+612\\
        3536&=612*5+476\\
        612&=476*1+136\\
        476&=136*3+68\\
        136&=68*68\\
    \end{align*}
    }
    \item [(c)] {$\lrp{26460,12600}$
    \begin{align*}
        26460&=12600*2+1260\\
        12600&=1260*10\\
    \end{align*}
    }
\end{itemize}
\end{multicols}
    \begin{multicols}{2}
        \begin{itemize}
    \item [(d)] {$\lrp{6540,1206}$
    \begin{align*}
        6540&=1206*5+510\\
        1206&=510*2+186\\
        510&=186*2+138\\
        186&=138*1+48\\
        138&=48*2+42\\
        48&=42*1+6\\
        42&=6*7\\
    \end{align*}
    }
    \item [(e)] {$\lrp{12091,8439}$
    \begin{align*}
        12091&=8439*1+3652\\
        8439&=3652*2+1135\\
        3652&=1135*3+247\\
        1135&=247*4+147\\
        247&=147*1+100\\
        147&=100*1+47\\
        100&=47*2+6\\
        47&=6*7+5\\
        6&=5*1+1\\
        5&=1*5\\
    \end{align*}
    }
\end{itemize}
\end{multicols}
\end{mdframed}
\newpage
%%%%%%%%%%%%%%%%%%%%%%%%%%%%%%%%%%%%%%%%%%%%%%%%%%%%%%%%%%%%%%%%%%%%%%%%%%%%%%%%
%%%%%%%%%%%%%%%%%%%%%%%%%%%%%%%%%%%%%%%%%%%%%%%%%%%%%%%%%%%%%%%%%%%%%%%%%%%%%%%%
%%%%%%%%%%%%%%%%%%%%%%%%%%%%%%%%%%%%%%%%%%%%%%%%%%%%%%%%%%%%%%%%%%%%%%%%%%%%%%%%
%%%%%%%%%%%%%%%%%%%%%%%%%%%%%%%%%%%%%%%%%%%%%%%%%%%%%%%%%%%%%%%%%%%%%%%%%%%%%%%%
\begin{mdframed}[style=darkQuesion]
    7.    For each part of Exercise 5, find integers $m$ and $n$ such that 
$\lrp{a,b}$ is expressed in the form $ma+nb$. 

\end{mdframed}

%%%%%%%%%%%%%%%%%%%%%%%%%%%%%%%%%%%%%%%%%%%%%%%%%%%%%%%%%%%%%%%%%%%%%%%%%%%%%%%%
\begin{mdframed}[style=darkAnswer,frametitle={Joe Starr}]
\begin{itemize}
    \item [(a)] {$\lrp{6643,2873}$\\
        $\lrp{6643}-16+\lrp{2873}37=13$
    }
    \item [(b)] {$\lrp{7684,4148}$\\
    $\lrp{7684}27+\lrp{4148}-50=68$
    }
    \item [(c)] {$\lrp{26460,12600}$
    $\lrp{26460}1+\lrp{12600}-2=1260$
    }
    \item [(d)] {$\lrp{6540,1206}$
   $\lrp{6540}-26+\lrp{1206}141=6$
    }
    \item [(e)] {$\lrp{12091,8439}$
    $\lrp{12091}1435+\lrp{8439}-2056=1$
    }
\end{itemize}
\end{mdframed}
\newpage
%%%%%%%%%%%%%%%%%%%%%%%%%%%%%%%%%%%%%%%%%%%%%%%%%%%%%%%%%%%%%%%%%%%%%%%%%%%%%%%%
%%%%%%%%%%%%%%%%%%%%%%%%%%%%%%%%%%%%%%%%%%%%%%%%%%%%%%%%%%%%%%%%%%%%%%%%%%%%%%%%
%%%%%%%%%%%%%%%%%%%%%%%%%%%%%%%%%%%%%%%%%%%%%%%%%%%%%%%%%%%%%%%%%%%%%%%%%%%%%%%%
%%%%%%%%%%%%%%%%%%%%%%%%%%%%%%%%%%%%%%%%%%%%%%%%%%%%%%%%%%%%%%%%%%%%%%%%%%%%%%%%
\begin{mdframed}[style=darkQuesion]
9.  let $a,b,c$ be integers such that $a+b+c=0$. Show that if $n$ is an integer
which is a divisor of two of the three integers, then it is also a divisor of 
the third. 

\end{mdframed}

%%%%%%%%%%%%%%%%%%%%%%%%%%%%%%%%%%%%%%%%%%%%%%%%%%%%%%%%%%%%%%%%%%%%%%%%%%%%%%%%
\begin{mdframed}[style=darkAnswer,frametitle={Joe Starr}]
Select $a,b,c\in \Z$ to satisfy $a+b+c=0$, WLOG let $n\in \Z$ such that 
$n\vert a$ and $n\vert b$. Since $\lrp{a+b}+c=0$ it must be that $\lrp{a+b}=-c$.
From here we must show $n\vert\lrp{a+b}$, or $a+b=nq$. Since $n\vert a$ and 
$n\vert b$ we may write $a=nq_1$ and $b=nq_2$, yielding, 
$nq_1+nq_2=n\lrp{q_1+q_2}=nq$ thus $n\vert c$, as desired.$\square$ 
\end{mdframed}
\newpage
%%%%%%%%%%%%%%%%%%%%%%%%%%%%%%%%%%%%%%%%%%%%%%%%%%%%%%%%%%%%%%%%%%%%%%%%%%%%%%%%
%%%%%%%%%%%%%%%%%%%%%%%%%%%%%%%%%%%%%%%%%%%%%%%%%%%%%%%%%%%%%%%%%%%%%%%%%%%%%%%%
%%%%%%%%%%%%%%%%%%%%%%%%%%%%%%%%%%%%%%%%%%%%%%%%%%%%%%%%%%%%%%%%%%%%%%%%%%%%%%%%
%%%%%%%%%%%%%%%%%%%%%%%%%%%%%%%%%%%%%%%%%%%%%%%%%%%%%%%%%%%%%%%%%%%%%%%%%%%%%%%%
\begin{mdframed}[style=darkQuesion]
13.  Show that if $n$ is any integer, then $\lrp{10n_3,5n+2}=1$
\end{mdframed}

%%%%%%%%%%%%%%%%%%%%%%%%%%%%%%%%%%%%%%%%%%%%%%%%%%%%%%%%%%%%%%%%%%%%%%%%%%%%%%%%
\begin{mdframed}[style=darkAnswer,frametitle={Joe Starr}]
We begin with the Euclidean algorithm, 
\begin{align*}
    10n+3&=\lrp{5n+2}1+\lrp{5n+1}\\
    5n+2&=\lrp{5n+1}1+1\\
\end{align*}
from here we have $\lrp{10n+3,5n+2}=\lrp{5n+2,5n+1}=1$, as desired.
\end{mdframed}
\newpage
%%%%%%%%%%%%%%%%%%%%%%%%%%%%%%%%%%%%%%%%%%%%%%%%%%%%%%%%%%%%%%%%%%%%%%%%%%%%%%%%
%%%%%%%%%%%%%%%%%%%%%%%%%%%%%%%%%%%%%%%%%%%%%%%%%%%%%%%%%%%%%%%%%%%%%%%%%%%%%%%%
%%%%%%%%%%%%%%%%%%%%%%%%%%%%%%%%%%%%%%%%%%%%%%%%%%%%%%%%%%%%%%%%%%%%%%%%%%%%%%%%
%%%%%%%%%%%%%%%%%%%%%%%%%%%%%%%%%%%%%%%%%%%%%%%%%%%%%%%%%%%%%%%%%%%%%%%%%%%%%%%%
\begin{mdframed}[style=darkQuesion]
15.  For what positive integers $n$ is it true that $\lrp{n,n+2}=2$? Prove your
claim.
\end{mdframed}

%%%%%%%%%%%%%%%%%%%%%%%%%%%%%%%%%%%%%%%%%%%%%%%%%%%%%%%%%%%%%%%%%%%%%%%%%%%%%%%%
\begin{mdframed}[style=darkAnswer,frametitle={Joe Starr}]
The conjecture is that the statement is true for even values of $n$.
We begin with rewriting $n$ in terms of $k$, $n=2k$the Euclidean algorithm, 
\begin{align*}
    \lrp{2k}+2&=\lrp{2k}1+\lrp{2}\\
    2k&=\lrp{2}{k}\\
\end{align*}
from here we have $\lrp{n+2,n}=\lrp{2k+2,2k}=2$, as desired.
\end{mdframed}
\newpage
%%%%%%%%%%%%%%%%%%%%%%%%%%%%%%%%%%%%%%%%%%%%%%%%%%%%%%%%%%%%%%%%%%%%%%%%%%%%%%%%
%%%%%%%%%%%%%%%%%%%%%%%%%%%%%%%%%%%%%%%%%%%%%%%%%%%%%%%%%%%%%%%%%%%%%%%%%%%%%%%%
%%%%%%%%%%%%%%%%%%%%%%%%%%%%%%%%%%%%%%%%%%%%%%%%%%%%%%%%%%%%%%%%%%%%%%%%%%%%%%%%
%%%%%%%%%%%%%%%%%%%%%%%%%%%%%%%%%%%%%%%%%%%%%%%%%%%%%%%%%%%%%%%%%%%%%%%%%%%%%%%%
\begin{mdframed}[style=darkQuesion]
17.  Show that the positive integer $k$ is the difference of two odd squares if 
and only if $k$ is divisible by $8$.
\end{mdframed}

%%%%%%%%%%%%%%%%%%%%%%%%%%%%%%%%%%%%%%%%%%%%%%%%%%%%%%%%%%%%%%%%%%%%%%%%%%%%%%%%
\begin{mdframed}[style=darkAnswer,frametitle={Joe Starr}]

We begin by writing $k=a^2-b^2$, since $a$ and $b$ are odd we can write, 
\begin{align*}
    a&=2r+1\\
    b&=2s+1
\end{align*} 
from here we have $q^2-b^2=4\lrp{r+s+1}\lrp{r-s}$. Since $k>0$ we must consider 
two cases $r-s=2m+1$ and $r-s=2m$. 
\begin{itemize}[align=left]
    \item [$r-s=2m$:]{\hspace{.5in}\newline
        In this case we have $q^2-b^2=4\lrp{r+s+1}2m=8\lrp{r+s+1}m$ and we are 
        done. 
    }
    \item [$r-s=2m+1$:]{\hspace{.5in}\newline
        In this case we have $r-s=2m+1$ and $r+s=r-s+2s=2m+1+2s$
        \begin{align*}
            q^2-b^2&=4\lrp{r+s+1}\lrp{2m+1}\\
            &=4\lrp{2m\lrp{r+s+1}+\lrp{r+s+1}}\\            
            &=4\lrp{\lrp{2mr+2ms+2m}+\lrp{r+s+1}}\\            
            &=4\lrp{2mr+2ms+2m+r+s+1}\\         
            &=4\lrp{2mr+2ms+2m+2m+1+2s+1}\\         
            &=4\lrp{2mr+2ms+2m+2m+2s+2}\\         
            &=8\lrp{mr+ms+m+m+s+1}\\         
        \end{align*}
        as desired.
    }
\end{itemize}

\end{mdframed}
\newpage
%%%%%%%%%%%%%%%%%%%%%%%%%%%%%%%%%%%%%%%%%%%%%%%%%%%%%%%%%%%%%%%%%%%%%%%%%%%%%%%%
%%%%%%%%%%%%%%%%%%%%%%%%%%%%%%%%%%%%%%%%%%%%%%%%%%%%%%%%%%%%%%%%%%%%%%%%%%%%%%%%
%%%%%%%%%%%%%%%%%%%%%%%%%%%%%%%%%%%%%%%%%%%%%%%%%%%%%%%%%%%%%%%%%%%%%%%%%%%%%%%%
%%%%%%%%%%%%%%%%%%%%%%%%%%%%%%%%%%%%%%%%%%%%%%%%%%%%%%%%%%%%%%%%%%%%%%%%%%%%%%%%
\subsection{Primes}
%%%%%%%%%%%%%%%%%%%%%%%%%%%%%%%%%%%%%%%%%%%%%%%%%%%%%%%%%%%%%%%%%%%%%%%%%%%%%%%%
%%%%%%%%%%%%%%%%%%%%%%%%%%%%%%%%%%%%%%%%%%%%%%%%%%%%%%%%%%%%%%%%%%%%%%%%%%%%%%%%
%%%%%%%%%%%%%%%%%%%%%%%%%%%%%%%%%%%%%%%%%%%%%%%%%%%%%%%%%%%%%%%%%%%%%%%%%%%%%%%%
%%%%%%%%%%%%%%%%%%%%%%%%%%%%%%%%%%%%%%%%%%%%%%%%%%%%%%%%%%%%%%%%%%%%%%%%%%%%%%%%
\begin{mdframed}[style=darkQuesion]
1. Find the prime factorizations of each of the following numbers, and use the 
them to compute the greatest common divisor and least common multiple of the 
given pairs of numbers. 
\begin{multicols}{3}
\begin{itemize}
    \item [(a)] {$\lrp{35,14}$
    }
    \item [(b)] {$\lrp{15,11}$
    }
    \item [(c)] {$\lrp{252,11}$
    }
    \item [(d)] {$\lrp{7684,4148}$
    }
    \item [(e)] {$\lrp{6643,2873}$
    }
\end{itemize}
\end{multicols}
\end{mdframed}

%%%%%%%%%%%%%%%%%%%%%%%%%%%%%%%%%%%%%%%%%%%%%%%%%%%%%%%%%%%%%%%%%%%%%%%%%%%%%%%%
\begin{mdframed}[style=darkAnswer,frametitle={Joe Starr}]
    \begin{multicols}{2}
    \begin{itemize}
        \item [(a)] {
        \begin{multicols}{2}
        $\lrp{35,14}$\\
        $35: 5,7$ \\
        $14: 2,7$ \\
        gcd: $7$ \\
        lcm: $70$ 
        \end{multicols}
        }
        \item [(b)] {
            \begin{multicols}{2}
        $\lrp{15,11}$ \\
        $15: 3,5$ \\
        $11: 11$ \\
        gcd: $1$ \\
        lcm: $165$ 
    \end{multicols}
        }
        \item [(c)] {
            \begin{multicols}{2}
        $\lrp{252,180}$ \\
        $252: 2,2,3,3,7$ \\
        $180: 2,2,3,3,5$ \\
        gcd: $36$ \\
        lcm: $1260$ 
    \end{multicols}
        }
        \item [(d)] {
            \begin{multicols}{2}
        $\lrp{7684,4148}$ \\
        $7684: 2,2,17,113$ \\
        $4148: 2,2,17,61$ \\
        gcd: $68$ \\
        lcm: $468724$ 
    \end{multicols}
        }
        \item [(e)] {
            \begin{multicols}{2}
        $\lrp{6643,2873}$ \\
        $6643: 7,13,73$ \\
        $2873: 13,13,17$ \\
        gcd: $13$ \\
        lcm: $1468103$ 
    \end{multicols}
        }
    \end{itemize}
\end{multicols}
\end{mdframed}
\newpage
%%%%%%%%%%%%%%%%%%%%%%%%%%%%%%%%%%%%%%%%%%%%%%%%%%%%%%%%%%%%%%%%%%%%%%%%%%%%%%%%
%%%%%%%%%%%%%%%%%%%%%%%%%%%%%%%%%%%%%%%%%%%%%%%%%%%%%%%%%%%%%%%%%%%%%%%%%%%%%%%%
%%%%%%%%%%%%%%%%%%%%%%%%%%%%%%%%%%%%%%%%%%%%%%%%%%%%%%%%%%%%%%%%%%%%%%%%%%%%%%%%
%%%%%%%%%%%%%%%%%%%%%%%%%%%%%%%%%%%%%%%%%%%%%%%%%%%%%%%%%%%%%%%%%%%%%%%%%%%%%%%%
\begin{mdframed}[style=darkQuesion]
2. US the sieve of Eratosthenes to find all prime numbers less than 200.
\end{mdframed}

%%%%%%%%%%%%%%%%%%%%%%%%%%%%%%%%%%%%%%%%%%%%%%%%%%%%%%%%%%%%%%%%%%%%%%%%%%%%%%%%
\begin{mdframed}[style=darkAnswer,frametitle={Joe Starr}]
    \begin{center}
        \begin{tabular}{| c | c | c | c | c | c | c | c | c | c |}
        \hline
        $ $ & $2$ & $3$ & $\msout{4}$ & $5$ & $\msout{6}$ & $7$ & $\msout{8}$ & $\msout{9}$ & $\msout{10}$ \\
        \hline
        $11$ & $\msout{12}$ & $13$ & $\msout{14}$ & $\msout{15}$ & $\msout{16}$ & $17$ & $\msout{18}$ & $19$ & $\msout{20}$ \\
        \hline
        $\msout{21}$ & $\msout{22}$ & $23$ & $\msout{24}$ & $\msout{25}$ & $\msout{26}$ & $\msout{27}$ & $\msout{28}$ & $29$ & $\msout{30}$ \\
        \hline
        $31$ & $\msout{32}$ & $\msout{33}$ & $\msout{34}$ & $\msout{35}$ & $\msout{36}$ & $37$ & $\msout{38}$ & $\msout{39}$ & $\msout{40}$ \\
        \hline
        $41$ & $\msout{42}$ & $43$ & $\msout{44}$ & $\msout{45}$ & $\msout{46}$ & $47$ & $\msout{48}$ & $\msout{49}$ & $\msout{50}$ \\
        \hline
        $\msout{51}$ & $\msout{52}$ & $53$ & $\msout{54}$ & $\msout{55}$ & $\msout{56}$ & $\msout{57}$ & $\msout{58}$ & $59$ & $\msout{60}$ \\
        \hline
        $61$ & $\msout{62}$ & $\msout{63}$ & $\msout{64}$ & $\msout{65}$ & $\msout{66}$ & $67$ & $\msout{68}$ & $\msout{69}$ & $\msout{70}$ \\
        \hline
        $71$ & $\msout{72}$ & $73$ & $\msout{74}$ & $\msout{75}$ & $\msout{76}$ & $\msout{77}$ & $\msout{78}$ & $79$ & $\msout{80}$ \\
        \hline
        $\msout{81}$ & $\msout{82}$ & $83$ & $\msout{84}$ & $\msout{85}$ & $\msout{86}$ & $\msout{87}$ & $\msout{88}$ & $89$ & $\msout{90}$ \\
        \hline
        $\msout{91}$ & $\msout{92}$ & $\msout{93}$ & $\msout{94}$ & $\msout{95}$ & $\msout{96}$ & $97$ & $\msout{98}$ & $\msout{99}$ & $\msout{100}$ \\
        \hline
        $101$ & $\msout{102}$ & $103$ & $\msout{104}$ & $\msout{105}$ & $\msout{106}$ & $107$ & $\msout{108}$ & $109$ & $\msout{110}$ \\
        \hline
        $\msout{111}$ & $\msout{112}$ & $113$ & $\msout{114}$ & $\msout{115}$ & $\msout{116}$ & $\msout{117}$ & $\msout{118}$ & $\msout{119}$ & $\msout{120}$ \\
        \hline
        $\msout{121}$ & $\msout{122}$ & $\msout{123}$ & $\msout{124}$ & $\msout{125}$ & $\msout{126}$ & $127$ & $\msout{128}$ & $\msout{129}$ & $\msout{130}$ \\
        \hline
        $131$ & $\msout{132}$ & $\msout{133}$ & $\msout{134}$ & $\msout{135}$ & $\msout{136}$ & $137$ & $\msout{138}$ & $139$ & $\msout{140}$ \\
        \hline
        $\msout{141}$ & $\msout{142}$ & $\msout{143}$ & $\msout{144}$ & $\msout{145}$ & $\msout{146}$ & $\msout{147}$ & $\msout{148}$ & $149$ & $\msout{150}$ \\
        \hline
        $151$ & $\msout{152}$ & $\msout{153}$ & $\msout{154}$ & $\msout{155}$ & $\msout{156}$ & $157$ & $\msout{158}$ & $\msout{159}$ & $\msout{160}$ \\
        \hline
        $\msout{161}$ & $\msout{162}$ & $163$ & $\msout{164}$ & $\msout{165}$ & $\msout{166}$ & $167$ & $\msout{168}$ & $\msout{169}$ & $\msout{170}$ \\
        \hline
        $\msout{171}$ & $\msout{172}$ & $173$ & $\msout{174}$ & $\msout{175}$ & $\msout{176}$ & $\msout{177}$ & $\msout{178}$ & $179$ & $\msout{180}$ \\
        \hline
        $181$ & $\msout{182}$ & $\msout{183}$ & $\msout{184}$ & $\msout{185}$ & $\msout{186}$ & $\msout{187}$ & $\msout{188}$ & $\msout{189}$ & $\msout{190}$ \\
        \hline
        $191$ & $\msout{192}$ & $193$ & $\msout{194}$ & $\msout{195}$ & $\msout{196}$ & $197$ & $\msout{198}$ & $199$ & $\msout{200}$\\
        \hline
        \end{tabular}
        \end{center}    
\end{mdframed}
\newpage
%%%%%%%%%%%%%%%%%%%%%%%%%%%%%%%%%%%%%%%%%%%%%%%%%%%%%%%%%%%%%%%%%%%%%%%%%%%%%%%%
%%%%%%%%%%%%%%%%%%%%%%%%%%%%%%%%%%%%%%%%%%%%%%%%%%%%%%%%%%%%%%%%%%%%%%%%%%%%%%%%
%%%%%%%%%%%%%%%%%%%%%%%%%%%%%%%%%%%%%%%%%%%%%%%%%%%%%%%%%%%%%%%%%%%%%%%%%%%%%%%%
%%%%%%%%%%%%%%%%%%%%%%%%%%%%%%%%%%%%%%%%%%%%%%%%%%%%%%%%%%%%%%%%%%%%%%%%%%%%%%%%
\begin{mdframed}[style=darkQuesion]
3. For each composite number $a$. with $4\leq a\leq 20$, find all positive 
numbers less than $a$ that are relatively prime to $a$.
\end{mdframed}

%%%%%%%%%%%%%%%%%%%%%%%%%%%%%%%%%%%%%%%%%%%%%%%%%%%%%%%%%%%%%%%%%%%%%%%%%%%%%%%%
\begin{mdframed}[style=darkAnswer,frametitle={Joe Starr}]
\begin{multicols}{1}
\begin{itemize}
    \item[$4:$]  {$2, 3$}
    \item[$6:$]  {$2, 3, 5$}
    \item[$8:$]  {$2, 3, 5, 7$}
    \item[$9:$]  {$2, 3, 4, 5, 7, 8$}
    \item[$10:$] {$2, 3, 5, 7, 9$}
    \item[$12:$] {$2, 3, 5, 7, 11$}
    \item[$14:$] {$2, 3, 5, 7, 9, 11, 13$}
    \item[$15:$] {$2, 3, 4, 5, 7, 8, 11, 13, 14$}
    \item[$16:$] {$2, 3, 5, 7, 9, 11, 13, 15$}
    \item[$18:$] {$2, 3, 5, 7, 11, 13, 17$}
    \item[$20:$] {$2, 3, 5, 7, 9, 11, 13, 17, 19$}
\end{itemize}
\end{multicols}
\end{mdframed}
\newpage
%%%%%%%%%%%%%%%%%%%%%%%%%%%%%%%%%%%%%%%%%%%%%%%%%%%%%%%%%%%%%%%%%%%%%%%%%%%%%%%%
%%%%%%%%%%%%%%%%%%%%%%%%%%%%%%%%%%%%%%%%%%%%%%%%%%%%%%%%%%%%%%%%%%%%%%%%%%%%%%%%
%%%%%%%%%%%%%%%%%%%%%%%%%%%%%%%%%%%%%%%%%%%%%%%%%%%%%%%%%%%%%%%%%%%%%%%%%%%%%%%%
%%%%%%%%%%%%%%%%%%%%%%%%%%%%%%%%%%%%%%%%%%%%%%%%%%%%%%%%%%%%%%%%%%%%%%%%%%%%%%%%
\begin{mdframed}[style=darkQuesion]
4. Find all positive integers less than 60 and relatively prime to $60$.
\end{mdframed}

%%%%%%%%%%%%%%%%%%%%%%%%%%%%%%%%%%%%%%%%%%%%%%%%%%%%%%%%%%%%%%%%%%%%%%%%%%%%%%%%
\begin{mdframed}[style=darkAnswer,frametitle={Joe Starr}]
\begin{itemize}
    \item[$60:$] {$2, 3, 5, 7, 11, 13, 17, 19, 23, 29, 31, 37, 41, 43, 47, 49, 53, 59 $}
\end{itemize}
\end{mdframed}
\newpage
%%%%%%%%%%%%%%%%%%%%%%%%%%%%%%%%%%%%%%%%%%%%%%%%%%%%%%%%%%%%%%%%%%%%%%%%%%%%%%%%
%%%%%%%%%%%%%%%%%%%%%%%%%%%%%%%%%%%%%%%%%%%%%%%%%%%%%%%%%%%%%%%%%%%%%%%%%%%%%%%%
%%%%%%%%%%%%%%%%%%%%%%%%%%%%%%%%%%%%%%%%%%%%%%%%%%%%%%%%%%%%%%%%%%%%%%%%%%%%%%%%
%%%%%%%%%%%%%%%%%%%%%%%%%%%%%%%%%%%%%%%%%%%%%%%%%%%%%%%%%%%%%%%%%%%%%%%%%%%%%%%%
\begin{mdframed}[style=darkQuesion]
\begin{itemize}
    \item [9. (a)] {For which $n\in \Z^{\text{+}}$ is $n^{3}-1$ a prime number?}
    \item [(b)] {For which $n\in \Z^{\text{+}}$ is $n^{3}+1$ a prime number?}
    \item [(c)] {For which $n\in \Z^{\text{+}}$ is $n^{2}-1$ a prime number?}
    \item [(d)] {For which $n\in \Z^{\text{+}}$ is $n^{2}+1$ a prime number?}
    \end{itemize}
\end{mdframed}

%%%%%%%%%%%%%%%%%%%%%%%%%%%%%%%%%%%%%%%%%%%%%%%%%%%%%%%%%%%%%%%%%%%%%%%%%%%%%%%%
\begin{mdframed}[style=darkAnswer,frametitle={Joe Starr}]
    \begin{itemize}
        \item [(a)] {For which $n\in \Z^{\text{+}}$ is $n^{3}-1$ a prime number?
        
        }
        \item [(b)] {For which $n\in \Z^{\text{+}}$ is $n^{3}+1$ a prime number?}
        \item [(c)] {For which $n\in \Z^{\text{+}}$ is $n^{2}-1$ a prime number?}
        \item [(d)] {For which $n\in \Z^{\text{+}}$ is $n^{2}+1$ a prime number?}
    \end{itemize}
\end{mdframed}

\newpage
%%%%%%%%%%%%%%%%%%%%%%%%%%%%%%%%%%%%%%%%%%%%%%%%%%%%%%%%%%%%%%%%%%%%%%%%%%%%%%%%
%%%%%%%%%%%%%%%%%%%%%%%%%%%%%%%%%%%%%%%%%%%%%%%%%%%%%%%%%%%%%%%%%%%%%%%%%%%%%%%%
%%%%%%%%%%%%%%%%%%%%%%%%%%%%%%%%%%%%%%%%%%%%%%%%%%%%%%%%%%%%%%%%%%%%%%%%%%%%%%%%
%%%%%%%%%%%%%%%%%%%%%%%%%%%%%%%%%%%%%%%%%%%%%%%%%%%%%%%%%%%%%%%%%%%%%%%%%%%%%%%%
\begin{mdframed}[style=darkQuesion]
11. Prove that $n^4+4^n$ is composite if $n>1$.
\end{mdframed}

%%%%%%%%%%%%%%%%%%%%%%%%%%%%%%%%%%%%%%%%%%%%%%%%%%%%%%%%%%%%%%%%%%%%%%%%%%%%%%%%
\begin{mdframed}[style=darkAnswer,frametitle={Joe Starr}]
We are presented with two potability's, $n$ is even or $n$ is odd.
\begin{itemize}[align=left]
    \item [$n$ even]{\hspace{.5in}\newline
        It's obvious that $n^4+4^n$ is an even not $2$ and can't be prime. 
    }
    \item [$n$ odd]{\hspace{.5in}\newline
    We begin by completing the square 
    \begin{align*}
        n^4+4^n&= n^4+4^n\\
        &=\lrp{n^2}^2+\lrp{2^n}^2\\
        &=\lrp{n^2+2^n}^2-2n^2 2^n
    \end{align*}
    We from here we observe that $2^n2=2^{n+1}$, since $n$ is odd $n+1$ is even 
    yielding $2^{n+1}=2^{2k}$. We can see we have a difference of squares
    \begin{align*}
        \lrp{n^2+2^n}^2-2n^2 2^n&=\lrp{n^2+2^n}^2-\lrp{2^nn}^2\\
        &=\lrp{n^2+2^n+2^nn}\lrp{n^2+2^n-2^nn}
    \end{align*}
    since we are restricted to $n>1$ we can see that both $\lrp{n^2+2^n+2^nn}>1$
    and $\lrp{n^2+2^n-2^nn}>1$ for all $n$. Making $n^4+4^n$ composite as 
    desired. 
    }
\end{itemize}
\end{mdframed}
\newpage
%%%%%%%%%%%%%%%%%%%%%%%%%%%%%%%%%%%%%%%%%%%%%%%%%%%%%%%%%%%%%%%%%%%%%%%%%%%%%%%%
%%%%%%%%%%%%%%%%%%%%%%%%%%%%%%%%%%%%%%%%%%%%%%%%%%%%%%%%%%%%%%%%%%%%%%%%%%%%%%%%
%%%%%%%%%%%%%%%%%%%%%%%%%%%%%%%%%%%%%%%%%%%%%%%%%%%%%%%%%%%%%%%%%%%%%%%%%%%%%%%%
%%%%%%%%%%%%%%%%%%%%%%%%%%%%%%%%%%%%%%%%%%%%%%%%%%%%%%%%%%%%%%%%%%%%%%%%%%%%%%%%
\begin{mdframed}[style=darkQuesion]
13. Let $a,b,c$  be positive integers, and let $d=\lrp{a,b}$. Since $d\vert a$, 
there exists an integer $h$ with $a=dh$. Show that $a\vert bc$, then $h\vert c$. 
\end{mdframed}

%%%%%%%%%%%%%%%%%%%%%%%%%%%%%%%%%%%%%%%%%%%%%%%%%%%%%%%%%%%%%%%%%%%%%%%%%%%%%%%%
\begin{mdframed}[style=darkAnswer,frametitle={Joe Starr}]
    
\end{mdframed}
\newpage
%%%%%%%%%%%%%%%%%%%%%%%%%%%%%%%%%%%%%%%%%%%%%%%%%%%%%%%%%%%%%%%%%%%%%%%%%%%%%%%%
%%%%%%%%%%%%%%%%%%%%%%%%%%%%%%%%%%%%%%%%%%%%%%%%%%%%%%%%%%%%%%%%%%%%%%%%%%%%%%%%
%%%%%%%%%%%%%%%%%%%%%%%%%%%%%%%%%%%%%%%%%%%%%%%%%%%%%%%%%%%%%%%%%%%%%%%%%%%%%%%%
%%%%%%%%%%%%%%%%%%%%%%%%%%%%%%%%%%%%%%%%%%%%%%%%%%%%%%%%%%%%%%%%%%%%%%%%%%%%%%%%
\begin{mdframed}[style=darkQuesion]
14. Show that $a\Z \cap b\Z=\lrb{a,b}\Z$.
\end{mdframed}

%%%%%%%%%%%%%%%%%%%%%%%%%%%%%%%%%%%%%%%%%%%%%%%%%%%%%%%%%%%%%%%%%%%%%%%%%%%%%%%%
\begin{mdframed}[style=darkAnswer,frametitle={Joe Starr}]
    
\end{mdframed}
\newpage
%%%%%%%%%%%%%%%%%%%%%%%%%%%%%%%%%%%%%%%%%%%%%%%%%%%%%%%%%%%%%%%%%%%%%%%%%%%%%%%%
%%%%%%%%%%%%%%%%%%%%%%%%%%%%%%%%%%%%%%%%%%%%%%%%%%%%%%%%%%%%%%%%%%%%%%%%%%%%%%%%
%%%%%%%%%%%%%%%%%%%%%%%%%%%%%%%%%%%%%%%%%%%%%%%%%%%%%%%%%%%%%%%%%%%%%%%%%%%%%%%%
%%%%%%%%%%%%%%%%%%%%%%%%%%%%%%%%%%%%%%%%%%%%%%%%%%%%%%%%%%%%%%%%%%%%%%%%%%%%%%%%
\begin{mdframed}[style=darkQuesion]
17. Let $a,b$ be nonzero integers. Prove $\lrp{a,b}=1$ if and only if 
$\lrp{a+b,ab}=1$. 
\end{mdframed}

%%%%%%%%%%%%%%%%%%%%%%%%%%%%%%%%%%%%%%%%%%%%%%%%%%%%%%%%%%%%%%%%%%%%%%%%%%%%%%%%
\begin{mdframed}[style=darkAnswer,frametitle={Joe Starr}]
    
\end{mdframed}
\newpage
%%%%%%%%%%%%%%%%%%%%%%%%%%%%%%%%%%%%%%%%%%%%%%%%%%%%%%%%%%%%%%%%%%%%%%%%%%%%%%%%
%%%%%%%%%%%%%%%%%%%%%%%%%%%%%%%%%%%%%%%%%%%%%%%%%%%%%%%%%%%%%%%%%%%%%%%%%%%%%%%%
%%%%%%%%%%%%%%%%%%%%%%%%%%%%%%%%%%%%%%%%%%%%%%%%%%%%%%%%%%%%%%%%%%%%%%%%%%%%%%%%
%%%%%%%%%%%%%%%%%%%%%%%%%%%%%%%%%%%%%%%%%%%%%%%%%%%%%%%%%%%%%%%%%%%%%%%%%%%%%%%%
\begin{mdframed}[style=darkQuesion]
18. Let $a,b$ be nonzero integers with $\lrp{a,b}=1$. Compute $\lrp{a+b,a-b}$. 
\end{mdframed}

%%%%%%%%%%%%%%%%%%%%%%%%%%%%%%%%%%%%%%%%%%%%%%%%%%%%%%%%%%%%%%%%%%%%%%%%%%%%%%%%
\begin{mdframed}[style=darkAnswer,frametitle={Joe Starr}]
    
\end{mdframed}
\newpage
%%%%%%%%%%%%%%%%%%%%%%%%%%%%%%%%%%%%%%%%%%%%%%%%%%%%%%%%%%%%%%%%%%%%%%%%%%%%%%%%
%%%%%%%%%%%%%%%%%%%%%%%%%%%%%%%%%%%%%%%%%%%%%%%%%%%%%%%%%%%%%%%%%%%%%%%%%%%%%%%%
%%%%%%%%%%%%%%%%%%%%%%%%%%%%%%%%%%%%%%%%%%%%%%%%%%%%%%%%%%%%%%%%%%%%%%%%%%%%%%%%
%%%%%%%%%%%%%%%%%%%%%%%%%%%%%%%%%%%%%%%%%%%%%%%%%%%%%%%%%%%%%%%%%%%%%%%%%%%%%%%%
\begin{mdframed}[style=darkQuesion]
19. Let $a$ and $b$ be positive integers, and let $m$ be an integer such that 
$ab=m\lrp{a,b}$. Without using the prime factorization theorem, prove that 
$\lrp{a,b}\lrb{a,b}=ab$ by verifying that $m$ satisfies the necessary properties
of $\lrb{a,b}$.
\end{mdframed}

%%%%%%%%%%%%%%%%%%%%%%%%%%%%%%%%%%%%%%%%%%%%%%%%%%%%%%%%%%%%%%%%%%%%%%%%%%%%%%%%
\begin{mdframed}[style=darkAnswer,frametitle={Joe Starr}]
    
\end{mdframed}
\newpage
%%%%%%%%%%%%%%%%%%%%%%%%%%%%%%%%%%%%%%%%%%%%%%%%%%%%%%%%%%%%%%%%%%%%%%%%%%%%%%%%
%%%%%%%%%%%%%%%%%%%%%%%%%%%%%%%%%%%%%%%%%%%%%%%%%%%%%%%%%%%%%%%%%%%%%%%%%%%%%%%%
%%%%%%%%%%%%%%%%%%%%%%%%%%%%%%%%%%%%%%%%%%%%%%%%%%%%%%%%%%%%%%%%%%%%%%%%%%%%%%%%
%%%%%%%%%%%%%%%%%%%%%%%%%%%%%%%%%%%%%%%%%%%%%%%%%%%%%%%%%%%%%%%%%%%%%%%%%%%%%%%%
\begin{mdframed}[style=darkQuesion]
20. A positive integer $a$ is called a square if $a=n^2$ for some $n\in \Z$. 
Show that the integer $a>1$ is a square if and only if every exponent in its 
prime factorization is even. 
\end{mdframed}

%%%%%%%%%%%%%%%%%%%%%%%%%%%%%%%%%%%%%%%%%%%%%%%%%%%%%%%%%%%%%%%%%%%%%%%%%%%%%%%%
\begin{mdframed}[style=darkAnswer,frametitle={Joe Starr}]
    
\end{mdframed}
\newpage
%%%%%%%%%%%%%%%%%%%%%%%%%%%%%%%%%%%%%%%%%%%%%%%%%%%%%%%%%%%%%%%%%%%%%%%%%%%%%%%%
%%%%%%%%%%%%%%%%%%%%%%%%%%%%%%%%%%%%%%%%%%%%%%%%%%%%%%%%%%%%%%%%%%%%%%%%%%%%%%%%
%%%%%%%%%%%%%%%%%%%%%%%%%%%%%%%%%%%%%%%%%%%%%%%%%%%%%%%%%%%%%%%%%%%%%%%%%%%%%%%%
%%%%%%%%%%%%%%%%%%%%%%%%%%%%%%%%%%%%%%%%%%%%%%%%%%%%%%%%%%%%%%%%%%%%%%%%%%%%%%%%
\begin{mdframed}[style=darkQuesion]
23. Let $p$ and $q$ be prime numbers. Prove that $pq+1$ is a square if and only 
if $p$ and $q$ are twin primes.
\end{mdframed}

%%%%%%%%%%%%%%%%%%%%%%%%%%%%%%%%%%%%%%%%%%%%%%%%%%%%%%%%%%%%%%%%%%%%%%%%%%%%%%%%
\begin{mdframed}[style=darkAnswer,frametitle={Joe Starr}]
    
\end{mdframed}
\newpage
%%%%%%%%%%%%%%%%%%%%%%%%%%%%%%%%%%%%%%%%%%%%%%%%%%%%%%%%%%%%%%%%%%%%%%%%%%%%%%%%
%%%%%%%%%%%%%%%%%%%%%%%%%%%%%%%%%%%%%%%%%%%%%%%%%%%%%%%%%%%%%%%%%%%%%%%%%%%%%%%%
%%%%%%%%%%%%%%%%%%%%%%%%%%%%%%%%%%%%%%%%%%%%%%%%%%%%%%%%%%%%%%%%%%%%%%%%%%%%%%%%
%%%%%%%%%%%%%%%%%%%%%%%%%%%%%%%%%%%%%%%%%%%%%%%%%%%%%%%%%%%%%%%%%%%%%%%%%%%%%%%%
\begin{mdframed}[style=darkQuesion]
26. Prove that if $a>1$, then there is a prime $p$ with $a<p\leq a!+1$.
\end{mdframed}

%%%%%%%%%%%%%%%%%%%%%%%%%%%%%%%%%%%%%%%%%%%%%%%%%%%%%%%%%%%%%%%%%%%%%%%%%%%%%%%%
\begin{mdframed}[style=darkAnswer,frametitle={Joe Starr}]
    
\end{mdframed}
\newpage
%%%%%%%%%%%%%%%%%%%%%%%%%%%%%%%%%%%%%%%%%%%%%%%%%%%%%%%%%%%%%%%%%%%%%%%%%%%%%%%%
%%%%%%%%%%%%%%%%%%%%%%%%%%%%%%%%%%%%%%%%%%%%%%%%%%%%%%%%%%%%%%%%%%%%%%%%%%%%%%%%
%%%%%%%%%%%%%%%%%%%%%%%%%%%%%%%%%%%%%%%%%%%%%%%%%%%%%%%%%%%%%%%%%%%%%%%%%%%%%%%%
%%%%%%%%%%%%%%%%%%%%%%%%%%%%%%%%%%%%%%%%%%%%%%%%%%%%%%%%%%%%%%%%%%%%%%%%%%%%%%%%
\begin{mdframed}[style=darkQuesion]
29. Show that $\log{2}/\log{3}$ is not a rational number. 
\end{mdframed}

%%%%%%%%%%%%%%%%%%%%%%%%%%%%%%%%%%%%%%%%%%%%%%%%%%%%%%%%%%%%%%%%%%%%%%%%%%%%%%%%
\begin{mdframed}[style=darkAnswer,frametitle={Joe Starr}]
    
\end{mdframed}


