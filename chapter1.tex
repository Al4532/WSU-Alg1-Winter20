\section{Integers}  
\subsection{Divisors}
%7 9 13 15 17
%%%%%%%%%%%%%%%%%%%%%%%%%%%%%%%%%%%%%%%%%%%%%%%%%%%%%%%%%%%%%%%%%%%%%%%%%%%%%%%%
%%%%%%%%%%%%%%%%%%%%%%%%%%%%%%%%%%%%%%%%%%%%%%%%%%%%%%%%%%%%%%%%%%%%%%%%%%%%%%%%
%%%%%%%%%%%%%%%%%%%%%%%%%%%%%%%%%%%%%%%%%%%%%%%%%%%%%%%%%%%%%%%%%%%%%%%%%%%%%%%%
%%%%%%%%%%%%%%%%%%%%%%%%%%%%%%%%%%%%%%%%%%%%%%%%%%%%%%%%%%%%%%%%%%%%%%%%%%%%%%%%
\begin{mdframed}[style=darkQuesion]
1.    Let $m,n.r.s\in \Z$. If $m^2+n^2=r^2+s^2=mr+ns$, prove that $m=r$ and 
$n=s$.
\end{mdframed}

%%%%%%%%%%%%%%%%%%%%%%%%%%%%%%%%%%%%%%%%%%%%%%%%%%%%%%%%%%%%%%%%%%%%%%%%%%%%%%%%
\begin{mdframed}[style=darkAnswer]
Joe Starr\\
We select $m,n.r.s\in \Z$, given $m^2+n^2=r^2+s^2=mr+ns$ which can write
as $m^2+n^2-mr-ns=r^2+s^2-mr-ns$. From here we can simplify:
\begin{align*}
m^2+n^2-mr-ns=r^2+s^2-mr-ns &\Rightarrow 
m\lrp{m-r}+n\lrp{n-s}=r\lrp{r-m}+s\lrp{s-n} \\
&\Rightarrow m\lrp{m-r}+n\lrp{n-s}-r\lrp{r-m}-s\lrp{s-n}=0 \\
&\Rightarrow m\lrp{m-r}+r\lrp{m-r}+n\lrp{n-s}+s\lrp{n-s}=0 \\
&\Rightarrow \lrp{m-r}\lrp{m+r}+\lrp{n-s}\lrp{n+s}=0 \\
\end{align*}
from here we can see that in order for $\lrp{m-r}\lrp{m+r}+\lrp{n-s}\lrp{n+s}=0$
to be true $m=r$ and $n=s$.
\end{mdframed}
\newpage
%%%%%%%%%%%%%%%%%%%%%%%%%%%%%%%%%%%%%%%%%%%%%%%%%%%%%%%%%%%%%%%%%%%%%%%%%%%%%%%%
%%%%%%%%%%%%%%%%%%%%%%%%%%%%%%%%%%%%%%%%%%%%%%%%%%%%%%%%%%%%%%%%%%%%%%%%%%%%%%%%
%%%%%%%%%%%%%%%%%%%%%%%%%%%%%%%%%%%%%%%%%%%%%%%%%%%%%%%%%%%%%%%%%%%%%%%%%%%%%%%%
%%%%%%%%%%%%%%%%%%%%%%%%%%%%%%%%%%%%%%%%%%%%%%%%%%%%%%%%%%%%%%%%%%%%%%%%%%%%%%%%
\begin{mdframed}[style=darkQuesion]
3.    Find the quotient and reminder when $a$ id divided by $b$. 
\begin{itemize}
    \item [a] {$a=99$, $b=17$}
    \item [b] {$a=-99$, $b=17$}
    \item [c] {$a=17$, $b=99$}
    \item [d] {$a=-1017$, $b=99$}
\end{itemize}

\end{mdframed}

%%%%%%%%%%%%%%%%%%%%%%%%%%%%%%%%%%%%%%%%%%%%%%%%%%%%%%%%%%%%%%%%%%%%%%%%%%%%%%%%
\begin{mdframed}[style=darkAnswer]
Joe Starr\\
\begin{itemize}
    \item [a] {$99=17q+r\Rightarrow q=5, r=14$}
    \item [b] {$-99=17q+r\Rightarrow q=-6, r=3$}
    \item [c] {$17=99q+r\Rightarrow q=0, r=17$}
    \item [d] {$-1017=99q+r\Rightarrow q=-11, r=72$}
\end{itemize}
\end{mdframed}

\newpage
%%%%%%%%%%%%%%%%%%%%%%%%%%%%%%%%%%%%%%%%%%%%%%%%%%%%%%%%%%%%%%%%%%%%%%%%%%%%%%%%
%%%%%%%%%%%%%%%%%%%%%%%%%%%%%%%%%%%%%%%%%%%%%%%%%%%%%%%%%%%%%%%%%%%%%%%%%%%%%%%%
%%%%%%%%%%%%%%%%%%%%%%%%%%%%%%%%%%%%%%%%%%%%%%%%%%%%%%%%%%%%%%%%%%%%%%%%%%%%%%%%
%%%%%%%%%%%%%%%%%%%%%%%%%%%%%%%%%%%%%%%%%%%%%%%%%%%%%%%%%%%%%%%%%%%%%%%%%%%%%%%%
\begin{mdframed}[style=darkQuesion]
5.    Use the Euclidean algorithm to find the following greatest common divisors
\begin{itemize}
    \item [a] {$\lrp{6643,2873}$}
    \item [b] {$\lrp{7684,4148}$}
    \item [c] {$\lrp{26460,12600}$}
    \item [d] {$\lrp{6540,1206}$}
    \item [e] {$\lrp{12091,8439}$}
\end{itemize}

\end{mdframed}

%%%%%%%%%%%%%%%%%%%%%%%%%%%%%%%%%%%%%%%%%%%%%%%%%%%%%%%%%%%%%%%%%%%%%%%%%%%%%%%%
\begin{mdframed}[style=darkAnswer]
Joe Starr\\
\begin{itemize}
    \item [a] {$\lrp{6643,2873}$
    \begin{align*}
        6643&=2873*2+897\\
        2873&=897*3+182\\
        897&=182*4+169\\
        182&=169*1+13\\
        169&=13*13\\
    \end{align*}
    }
    \item [b] {$\lrp{7684,4148}$
    \begin{align*}
        7684&=4148*1+3536\\
        4148&=3536*1+612\\
        3536&=612*5+476\\
        612&=476*1+136\\
        476&=136*3+68\\
        136&=68*68\\
    \end{align*}
    }
    \item [c] {$\lrp{26460,12600}$
    \begin{align*}
        26460&=12600*2+1260\\
        12600&=1260*10\\
    \end{align*}
    }
    \item [d] {$\lrp{6540,1206}$
    \begin{align*}
        6540&=1206*5+510\\
        1206&=510*2+186\\
        510&=186*2+138\\
        186&=138*1+48\\
        138&=48*2+42\\
        48&=42*1+6\\
        42&=6*7\\
    \end{align*}
    }
    \item [e] {$\lrp{12091,8439}$
    \begin{align*}
        12091&=8439*1+3652\\
        8439&=3652*2+1135\\
        3652&=1135*3+247\\
        1135&=247*4+147\\
        247&=147*1+100\\
        147&=100*1+47\\
        100&=47*2+6\\
        47&=6*7+5\\
        6&=5*1+1\\
        5&=1*5\\
    \end{align*}
    }
\end{itemize}
\end{mdframed}
\newpage
%%%%%%%%%%%%%%%%%%%%%%%%%%%%%%%%%%%%%%%%%%%%%%%%%%%%%%%%%%%%%%%%%%%%%%%%%%%%%%%%
%%%%%%%%%%%%%%%%%%%%%%%%%%%%%%%%%%%%%%%%%%%%%%%%%%%%%%%%%%%%%%%%%%%%%%%%%%%%%%%%
%%%%%%%%%%%%%%%%%%%%%%%%%%%%%%%%%%%%%%%%%%%%%%%%%%%%%%%%%%%%%%%%%%%%%%%%%%%%%%%%
%%%%%%%%%%%%%%%%%%%%%%%%%%%%%%%%%%%%%%%%%%%%%%%%%%%%%%%%%%%%%%%%%%%%%%%%%%%%%%%%
\begin{mdframed}[style=darkQuesion]
    7.    For each part of Exercise 5, find integers $m$ and $n$ such that 
$\lrp{a,b}$ is expressed in the form $ma+nb$. 

\end{mdframed}

%%%%%%%%%%%%%%%%%%%%%%%%%%%%%%%%%%%%%%%%%%%%%%%%%%%%%%%%%%%%%%%%%%%%%%%%%%%%%%%%
\begin{mdframed}[style=darkAnswer]
    Joe Starr\\
\begin{itemize}
    \item [a] {$\lrp{6643,2873}$
    
    }
    \item [b] {$\lrp{7684,4148}$
    
    }
    \item [c] {$\lrp{26460,12600}$
    
    }
    \item [d] {$\lrp{6540,1206}$
   
    }
    \item [e] {$\lrp{12091,8439}$
    
    }
\end{itemize}
\end{mdframed}
\newpage
%%%%%%%%%%%%%%%%%%%%%%%%%%%%%%%%%%%%%%%%%%%%%%%%%%%%%%%%%%%%%%%%%%%%%%%%%%%%%%%%
%%%%%%%%%%%%%%%%%%%%%%%%%%%%%%%%%%%%%%%%%%%%%%%%%%%%%%%%%%%%%%%%%%%%%%%%%%%%%%%%
%%%%%%%%%%%%%%%%%%%%%%%%%%%%%%%%%%%%%%%%%%%%%%%%%%%%%%%%%%%%%%%%%%%%%%%%%%%%%%%%
%%%%%%%%%%%%%%%%%%%%%%%%%%%%%%%%%%%%%%%%%%%%%%%%%%%%%%%%%%%%%%%%%%%%%%%%%%%%%%%%
\begin{mdframed}[style=darkQuesion]
9.  let $a,b,c$ be integers such that $a+b+c=0$. Show that if $n$ is an integer
which is a divisor of two of the three integers, then it is also a divisor of 
the third. 

\end{mdframed}

%%%%%%%%%%%%%%%%%%%%%%%%%%%%%%%%%%%%%%%%%%%%%%%%%%%%%%%%%%%%%%%%%%%%%%%%%%%%%%%%
\begin{mdframed}[style=darkAnswer]
Joe Starr\\
Select $a,b,c\in \Z$ to satisfy $a+b+c=0$, WLOG let $n\in \Z$ such that 
$n\vert a$ and $n\vert b$. Since $\lrp{a+b}+c=0$ it must be that $\lrp{a+b}=-c$.
From here we must show $n\vert\lrp{a+b}$, or $a+b=nq$. Since $n\vert a$ and 
$n\vert b$ we may write $a=nq_1$ and $b=nq_2$, yielding, 
$nq_1+nq_2=n\lrp{q_1+q_2}=nq$ thus $n\vert c$, as desired.$\square$ 
\end{mdframed}